\newpage
\chapter*{Eidesstattliche Erklärung}
\chaptermark{Eidesstattliche Erklärung}
\addcontentsline{}{}{Eidesstattliche Erklärung}

Hiermit erkläre ich, Robin Buhlmann, dass ich die vorliegende Bachelorarbeit selbstständig und ohne unerlaubte Hilfe angefertigt, andere als die angegebenen Quellen und Hilfsmittel nicht benutzt und die den benutzten Quellen wörtlich oder inhaltlich entnommenen Stellen als solche kenntlich gemacht habe.

Die Arbeit wurde bisher in gleicher oder ähnlicher Form keiner anderen Prüfungsbehörde vorgelegt und auch nicht veröffentlicht.


\vspace{1cm}

Friedrichsdorf, den

\vspace{1cm}

------------------------------------\hspace{6cm}------------------------------------\\
Datum \hspace{9,4cm} Unterschrift

\newpage

\chapter*{Danksagung}
\chaptermark{Danksagung}
\addcontentsline{}{}{Danksagung}
\setlength{\parindent}{0em}
\setlength{\parskip}{1em}

Hiermit möchte ich mich insbesondere bei meiner Familie für die Unterstützung während meines Studiums, sowie dieser Arbeit bedanken. Ich danke Professor Godehardt, dass dieser sich trotz vollem Terminplan auf dieses Thema eingelassen hat, sowie für die Hilfestellung bei der Themenfindung und für die Betreuung dieser Arbeit. Auch möchte ich Professor Baun danken, dass er sich als Zweitprüfer bereit erklärt hat.\par

Besonderer Dank gilt natürlich meiner Kaffeemaschine, welche das Erstellen dieser Arbeit bedauerlicherweise nicht überstanden hat.\par

\begin{displayquote} 
\textit{"
Die wohl absurdeste Art aller Netzwerke sind die Computernetzwerke. Diese Werke werden von ständig rechnenden Computern vernetzt und niemand weiß genau warum sie eigentlich existieren. Wenn man den Gerüchten Glauben schenken darf, dann soll es sich hierbei um werkende Netze handeln die das Arbeiten und das gesellschaftliche Miteinander fordern und fördern sollen. Großen Anteil daran soll ein sogenanntes Internet haben, dass wohl sehr weit verbreitet sein soll. Viele Benutzer des Internets leben allerdings das genetzwerkte Miteinander so sehr aus, dass das normale Miteinander nahezu komplett vernachlässigt wird (vgl. World of Warcraft)."}\par
\hfill
 -- Netzwerke -- www.stupidedia.org
\end{displayquote}

\vspace{-55pt}
\begin{singlespace}
\chapter*{Zusammenfassung}
\chaptermark{Abstract}
\addcontentsline{}{}{Abstract}

Peer-To-Peer-Netzwerkinfrastrukturen sind bei der Entwicklung von Mehrspieler-Spielen -- aufgrund von weniger Hardwarebedarf zur Datenübertragung zwischen Nutzern -- in der Regel kosteneffizienter als Client-Server-Modelle. Mit dem Lebensende des Adobe Flash Players im Dezember 2020 ist WebRTC nun (Stand 2021) die einzige Browserbasierte Peer-To-Peer Technologie, welche ohne Plugins verwendbar ist. Diese ist jedoch im Bereich der Browserbasierten Mehrspieler-Spieleentwicklung noch nicht weit verbreitet. Mit der Veröffentlichung von WebRTC als offizieller Web-Standard durch das W3C ist WebRTC nun bereit für weit verbreitete Anwendung, auch in Bereichen außerhalb der Video- und Audiokommunikation.\par

Diese Arbeit befasst sich mit der Anwendbarkeit von WebRTC zur Entwiclung von Browserbasierten Mehrspieler-Brettspielen  im Vergleich zu \glqq{}traditionellen\grqq{} Client-Server-Architekturen. Dabei wird eine WebRTC-Netzwerkinfrastruktur konzipiert. Auf dieser wird das Brettspiel \glqq{}Mensch Ärgere Dich Nicht\grqq{} prototypisch entwickelt und aufgesetzt.\par 

\let\clearpage\relax
\chapter*{Abstract}
Peer-To-Peer networks are widely used for the development of multiplayer video games. Peer-To-Peer networks tend to be more cost-efficient than client-server networks, due to less computing-resources required to forward data between players. With the end-of-life of the Adobe Flash Player in December 2020, WebRTC is now the only peer-to-peer technology, which can be used out-of-the-box in most browsers without requiring additional plugins. WebRTC is currently (Q1 2021) not widely used for the development of browser-based multiplayer video games. Especially with the release of the WebRTC-Recommendation by the W3C in January 2021, WebRTC is now a fully fledged web-standard, ready for wide-spread use in areas besides video- and audiocommunications.\par

This thesis evaluates the usability of WebRTC for the development of browser-based multiplayer board games in contrast to more \grqq{}traditional\grqq{} client-server network models. A WebRTC network infrastructure is deployed. Based on this, a prototype of the popular german board game \grqq{}Mensch ärgere Dich nicht\grqq{} is developed and deployed.\par
\end{singlespace}
