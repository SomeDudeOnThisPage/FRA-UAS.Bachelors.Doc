Ziel dieser Arbeit war das Aufsetzen einer prototypischen WebRTC-Netzwerkinfrastruktur, welche das Spielen eines Brettspiels ermöglicht. Zudem wurde die Anwendbarkeit von \acs{WebRTC} für browserbasierte Spiele mit standartmäßig verwendeten Client-Server-Architekturen verglichen.\par

Rückblickend wurde mit der Implementierung das Ziel, ein Brettspiel mithilfe einer WebRTC-Netzwerkinfrastruktur zu implementieren, erreicht. Zudem wurden Ansätze zur Vermeidung von Betrug diskutiert. Diese sind jedoch in diesem Fall von geringerer Bedeutung als anfänglich geschätzt, da Spieler bei diesem Gesellschaftsspiel -- welches in der Regel von untereinander bereits bekannten Personen, und nicht Fremden gespielt wird -- keinen Anreiz haben, zu betrügen.\par

Insbesondere dieser Betrugsaspekt bietet einen Ansatz für zukünftige Arbeiten. Peer-To-Peer-Spiele sind generell anfälliger für Betrug als Spiele, welche einen zentralen Server zur Synchronisation der Spieldaten nutzen. Betrug wird weiterhin durch den -- durch die Implementierung im Browser bedingten -- offenliegenden Quellcode erleichtert. Hier müssen existierende, weit verbreitete Anti-Betrugsmaßnahmen für \acs{P2P}-Spiele -- wie zum Beispiel das Lockstep-Protokoll oder Commitment-Verfahren -- mitunter erneut evaluiert werden. Ein weiterer Ansatz ist die Konzeption, das Design und die Implementierung eines vollends betrugssicheren \acs{P2P}-Spiels im Browser, welches auf der in dieser Arbeit implementierten \acs{WebRTC}-Netzwerkstruktur basiert.\par

Ein weiterer Ansatz für zukünftige Arbeit ist die Verwendung von hybriden Netzwerkinfrastrukturen. Neben Telepräsenz kann WebRTC für ergänzende Echtzeit-Features in Brettspielen -- zum Beispiel das Zeichnen auf einem Spielbrett oder Drag-and-Drop von Spielfiguren, wobei die Position der Spielfigur zwischen Spielern geteilt wird -- verwendet werden.\par

Letztendlich wurde deutlich, dass WebRTC eine flexible, wenn auch komplexere Alternative zu Client-Server-Modellen darstellt, welche im Bereich der Browserbasierten Brettspielentwicklung Potential hat. Der größte Nachteil der Anwendung von WebRTC ist, wie bei allen \acs{P2P}-Technologien auch, der Aspekt des Betrugs. Ist dieser nicht relevant, so bietet WebRTC eine kostengünstige Alternative zu Client-Server-Architekturen.\par

Mit dem Lebensende des Adobe Flash Players, und der Veröffentlichung als Web-Standard, ist WebRTC jetzt vollkommen Bereit für weitverbreitete Verwendung -- auch im Bereich der Browserbasierten Spieleentwicklung.\par