\appendix
\chapter{Anhang -- Peer.js}
\lstset{language=js, style=STYLE_CODE_JS}
\begin{singlespace}
\begin{lstlisting}[]
export default function Peer(id, iceServers) {
  this.id = id;
  this.callbacks = {};

  this.rtcConfiguration = { iceServers : iceServers };
  this.connections = {};

  this.signal = null;   // callback to send a signaling message to a remote peer
}

Peer.prototype.closeConnections = function() {
  Object.values(this.connections).forEach((connection) => connection.close());
}

Peer.prototype.closeConnection = function(remotePeerId) {
  if (this.connections[remotePeerId]) {
    this.connections[remotePeerId].close();
    delete this.connections[remotePeerId];
  }
}

Peer.prototype.setICETransportPolicy = function(policy) {
  if (policy !== 'relay' && policy !== 'all') {
    console.error('ICE transport policy must be of values [relay, all]');
    return;
  }

  this.rtcConfiguration.iceTransportPolicy = policy;
}

Peer.prototype.setICEServers = function(servers) {
  this.rtcConfiguration.iceServers = servers;
}

Peer.prototype.setSignalingCallback = function(cb) {
  this.signal = cb;
}

Peer.prototype._receiveMessage = function(e) {
  const message = JSON.parse(e.data);

  if (message.event && this.callbacks[message.event]) {
    this.callbacks[message.event](...message.data, message.src);
  }
}

Peer.prototype._dataChannelOpen = function(remotePeerId) {
  console.log(`[${remotePeerId}] data channel open`);

  // check if all channels of all connections are open
  if (Object.values(this.connections).every((connection) => {
    return connection.dc.readyState === 'open'
  })) {
    // if a callback for this event exists, run it
    if (this.callbacks['onDataChannelsOpen']) this.callbacks['onDataChannelsOpen']();
  }
}

Peer.prototype._createConnection = function(remotePeerId) {
  const connection = new RTCPeerConnection(this.rtcConfiguration);
  connection.dc = {}; // list of data channels belonging to this connection

  // setup a reliable and ordered data channel, store in connection object
  const channel = connection.createDataChannel('game', {
    negotiated : true,
    id : 0,
    maxRetransmits : null,    // no maximum number of retransmits
    ordered : true            // force ordered package retrieval
  });
  channel.onmessage = (e) => this._receiveMessage(e);
  channel.onopen = () => this._dataChannelOpen(channel.label, remotePeerId);

  connection.dc = channel;

  connection.onsignalingstatechange = () => {
    console.log(`[${remotePeerId}] signaling state '${connection.signalingState}'`);
  }

  connection.onicecandidate = (e) => {
    this.signal(this._createSignal('ice-candidate', e.candidate, remotePeerId));
  }

  connection.onicecandidateerror = (e) => {
    if (e.errorCode >= 300 && e.errorCode <= 699) {
      console.warn('STUN-Server could not be reached or threw error.', e)
    } else if (e.errorCode >= 700 && e.errorCode <= 799) {
      console.warn('TURN-Server could not be reached.', e)
    }
  }

  return connection;
}

Peer.prototype._createSignal = function(type, data, target) {
  return {type : type, src : this.id, target : target, data : data}
}

Peer.prototype.connect = function(remotePeerId) {
  const connection = this._createConnection(remotePeerId, true);
  this.connections[remotePeerId] = connection;

  connection.createOffer().then((offer) => {
    this.signal(this._createSignal('offer', offer, remotePeerId));
    return connection.setLocalDescription(offer);
  }).catch((e) => console.error(e));
}

Peer.prototype.onsignal = function(e) {
  if (e.target && e.target === this.id) {
    switch(e.type) {
      case 'offer': // on offer, create our answer, set our local description and signal the remote peer
        const connection = this._createConnection(e.src);
        this.connections[e.src] = connection;

        connection.setRemoteDescription(e.data).then(() => {
          console.log('remote description set');
          return connection.createAnswer();
        }).then((answer) => {
          this.signal(this._createSignal('answer', answer, e.src));
          return connection.setLocalDescription(answer);
        }).catch((e) => console.error(e));
        break;
      case 'answer': // on answer, set our remote description
        this.connections[e.src].setRemoteDescription(e.data).then(() => console.log('remote description set')).catch(e => console.error(e));
        break;
      case 'ice-candidate': // on ice candidate, add the ice candidate to the corresponding connection
        console.log('ice-candidate', e.data);
        this.connections[e.src].addIceCandidate(e.data).then();
        break;
    }
  }
}

Peer.prototype.on = function(e, cb) {
  this.callbacks[e] = cb;
}

// TODO: not needed for mesh-architecture - maybe remove?
Peer.prototype.emit = function(target, e, ...args) { // function for the master peer to relay a targeted message
  if (this.connections[target]) {
    const data = JSON.stringify({src : this.id, event: e, data: args});
    this.connections[target].dc.send(data);
  }
}

// TODO: not needed for mesh-architecture - maybe remove?
Peer.prototype.relay = function(exclude,packet) { // function for the master peer to relay a message
  Object.keys(this.connections).forEach((key) => {
    if (key !== exclude) {
      this.connections[key].dc.send(packet);
    }
  });
}

Peer.prototype.broadcast = function(e, ...args) { // function for any peer to broadcast a message
  Object.values(this.connections).forEach((connection) => {
    const data = JSON.stringify({src : this.id, event: e, data: args});
    connection.dc.send(data);
  });
}
\end{lstlisting}
\end{singlespace}

\chapter{Anhang -- Server.js}
\lstset{language=js, style=STYLE_CODE_JS}
\begin{singlespace}
\begin{lstlisting}[]
const express = require('express');
const config = require('./config.json');
const utils = require('./utils.js');

const app = express();

const server = require('http').Server(app);
const io = require('socket.io')(server);
app.use(express.static(config.server.static));

// serve files
app.get('/', (req, res) => {
  res.status(200);
  res.sendFile(`${__dirname}${config.server.index}`);
});

app.get(`/game/*/`, (req, res) => {
  res.status(200);
  res.sendFile(`${__dirname}${config.server.game}`);
});

server.listen(config.server.listeningPort);

// room = {
//   id = <four-letter room ID>
//   started = true | false
//   host = <socket-ID of the host>
//   players = []
// }
const rooms = {};

// we don't want to store the socket-ids of players inside the player array of the room, because we don't want to send
// them to the players when they connect. But we need to find the socket-id of a peer when signaling, so this object
// mapping peerID -> socketID of every player exists.
const sockets = {};

io.sockets.on('connection', (socket) => {
  socket.on('game-room-create', () => {
    const id = utils.generateRoomID(rooms);
    rooms[id] = {
      id : id, // easier to identify room by player
      players : [null, null, null, null],
      started : false,
      host : null
    };

    // if no client joins within a minute, destroy the room again (something went wrong on the clients' side, as the redirect didn't go through)
    setTimeout(() => {
      if (rooms[id] && rooms[id].players.length === 0) {
        console.log('DELETED ROOM DUE TO INACTIVITY TIMEOUT');
        delete rooms[id];
      }
    }, 60000);

    // const room = manager.createRoom(app, 'Test');
    socket.emit('game-room-created', id);
  });

  const PLAYER_SLOT_PRIORITY = [0, 2, 1, 3]; // Als erstes gegenueberliegende Farben fuellen
  socket.on('game-room-join', (roomID) => {
    const room = rooms[roomID];

    if (!room) return socket.emit('game-room-join-failed', 'no such room');
    if (room.players.every((p) => p !== null)) return socket.emit('game-room-join-failed', 'room full');

    for (let i = 0; i < 4; i++) {
      if (!room.players[PLAYER_SLOT_PRIORITY[i]]) {
        const peerID = utils.uuid4();
        const color = PLAYER_SLOT_PRIORITY[i];
        console.log('NEW PLAYER COLOR INDEX', color);

        socket.join(roomID); // broadcasting for a subset of sockets is done via socket.io rooms
        sockets[peerID] = socket.id; // sockets mapped to peer-id for signaling (can't be in player object, because we don't want to send that)
        room.players[color] = {peerID : peerID, color : color};
        room.seed = Math.random();

        if (!room.host) {
          room.host = socket.id;
        }

        socket.emit('game-room-joined', room.players, room.started, room.seed, peerID, utils.generateTURNCredentials(socket.id), room.host === socket.id);
        socket.to(roomID).emit('game-room-client-joining', room.seed, peerID, color);
        break;
      }
    }
  });

  socket.on('start-game', (roomId) => {
    const room = rooms[roomId];

    if (room && !room.started && room.host === socket.id) {
      room.started = true;

      // Seed fuer die Zufallsfunktion
      room.seed = Math.random();

      // Seed an alle Spieler des Raums senden
      io.to(roomId).emit('game-start', room.seed);
    }
  });

  socket.on('game-room-kick-player', (roomId, index) => {
    const room = rooms[roomId];

    if (room && socket.id === room.host && room.players[index]) {
      io.to(roomId).emit('game-room-kick-player', room.players[index].peerID, index);
      room.players[index] = null;
    }
  });

  socket.on('signal', (roomID, targetID, e) => {
    const room = rooms[roomID];

    if (room) {
      const target = room.players.find((player) => player && player.peerID === targetID);
      if (target) {
        socket.to(sockets[target.peerID]).emit('signal', e);
      }
    }
  });

  socket.on('disconnecting', () => {
    // the following search operations aren't super efficient when the player count gets large,
    // but it's for the sake of having less LOC to clutter up the document.
    // if this was an actual production application, there should likely be a map of
    // socketID -> {peerID, roomID} for each socket connected, to have O(1) lookup times
    const peerID = Object.keys(sockets).find((peerID) => sockets[peerID] === socket.id);
    const room = Object.values(rooms).find((room) => room.players.find((player) => player && player.peerID === peerID));

    if (room) {
      const leaver = room.players.find((player) => player && player.peerID === peerID);
      console.log('LEAVER PLAYER COLOR INDEX', leaver.color)
      if (leaver) {
        room.players[leaver.color] = null;

        // Host migration
        if (socket.id === room.host && room.players.length > 0) {
          const newHost = room.players.find((p) => p !== null);
          if (newHost) {
            room.host = sockets[newHost.peerID];
            io.to(room.id).emit('game-room-host-migration', newHost.peerID);
          }
        }

        if (room.players.length > 0) {
          socket.to(room.id).emit('game-room-client-leaving', leaver.peerID, leaver.color); // notify remaining
        } else {
          delete rooms[room.id]; // remove the room when the last player left the room
        }
      }
    }
  });
});
\end{lstlisting}
\end{singlespace}


\chapter{Anhang -- game.js}
\lstset{language=js, style=STYLE_CODE_JS}
\begin{singlespace}
\begin{lstlisting}[]
import Peer from './network/Peer.js';
import {Game} from './game/Game.js';
import Player from './game/Player.js';
import * as utils from './game/utils.js';

const GAME_SERVER_CONNECTION_DEV = 'http://localhost:3000';
const GAME_SERVER_CONNECTION_PROD = 'http://20.56.95.156:3000'; // this should probably be injected by express middleware...

const DATA_CHANNELS = [{ // reliable, ordered data channel
  label : 'game',
  rtcDataChannelConfig : {
    maxRetransmits : null,    // no maximum number of retransmits
    ordered : true            // force ordered package retrieval
  }
}];

$(window).on('load', () => {
  const socket = io(GAME_SERVER_CONNECTION_DEV);
  const roomID = window.location.pathname.split('/')[2];

  socket.on('connect', () => {
    socket.emit('game-room-join', roomID); // Raum beitreten
  });

  socket.on('game-room-joined', (players, started, seed, peerID, iceServers, isHost) => {
    let localPlayerIsHost = isHost;

    const peer = new Peer(peerID, iceServers, DATA_CHANNELS); // erstellen des Peers
    peer.setSignalingCallback((data) => socket.emit('signal', roomID, data.target, data)); // Signalisierung VOM Peer weiterleiten
    socket.on('signal', (e) => peer.onsignal(e)); // Signalisierung ZUM Peer weiterleiten

    const game = new Game($('#maedn'), peer, DATA_CHANNELS[0].label);
    socket.on('game-start', (seed) => game.start(seed));
    if (started) {
      game.seed(seed);
      game.paused = true;
    }
    game.render();

    socket.on('game-room-client-joining', (seed, peerID, color) => {
      game.paused = true;
      game.seed(seed);
      game.addPlayer(new Player(color, 'TestPlayer', false));
      utils.uiEnablePlayerKickButtons(game.players, localPlayerIsHost, roomID, socket);
    });

    socket.on('game-room-client-leaving', (peerID, color) => {
      peer.closeConnection(peerID);
      game.removePlayer(color);
      utils.uiEnablePlayerKickButtons(game.players, localPlayerIsHost, roomID, socket);
    });

    socket.on('game-room-host-migration', (newHostPeerID) => {
      console.log('HOST MIGRATION', newHostPeerID === peerID, newHostPeerID, peerID);
      localPlayerIsHost = newHostPeerID === peerID;
      $('#start-game').prop('disabled', newHostPeerID !== peerID).click(() => socket.emit('start-game', roomID));
      $('#end-game').prop('disabled', newHostPeerID !== peerID).click(() => socket.emit('end-game', roomID));
    });

    socket.on('game-room-kick-player', (peerID, color) => {
      if (game.players[color]) {
        if (game.players[color].isLocalPlayer) {
          // the server removes us when we are kicked, NOT when we disconnect
          // so leave this page normally
          window.location.href = '/';
        } else {
          peer.closeConnection(peerID);
          game.removePlayer(color);
          utils.uiEnablePlayerKickButtons(game.players, localPlayerIsHost, roomID, socket);
        }
      }
    });

    peer.on('onDataChannelsOpen', () => {
      if (localPlayerIsHost) {
        peer.broadcast('gamestate', game.gamestate);
      }
      // unpause whenever ALL dcs to other peers are open (meaning also new ones)
      game.paused = false;
    });

    peer.on('gamestate', (gamestate) => {
      game.setGamestate(gamestate);
      game.paused = false; // unpause when receiving game state
    });

    players.forEach((player) => { // connect to other peers
      if (!player) return;
      player.peerID === peerID ? game.localPlayerColor = player.color : peer.connect(player.peerID);
      game.addPlayer(new Player(player.color, 'TestPlayer', game.localPlayerColor === player.color));
    });

    utils.uiEnablePlayerKickButtons(game.players, localPlayerIsHost, roomID, socket);

    // utility stuff, don't go over this in document
    $('#start-game').prop('disabled', !localPlayerIsHost).click(() => socket.emit('start-game', roomID, socket.id));
    $('#end-game').prop('disabled', !localPlayerIsHost).click(() => socket.emit('end-game', roomID, socket.id));
  });
});
\end{lstlisting}
\end{singlespace}