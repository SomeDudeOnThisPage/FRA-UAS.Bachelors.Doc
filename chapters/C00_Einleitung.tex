Am 26. Januar 2021 veröffentlichte das \ac{W3C} die \ac{WebRTC} Recommendation. Eine Recommendation des \acs{W3C} ist ein offizieller Web-Standard in seiner -- bezüglich der zentralen Funktionalität -- finalen Form. \acs{WebRTC} ist eine Peer-To-Peer Web-Technologie, und wird primär für Webbrowserbasierte Echtzeit-Anwendungen wie Audio- und Videokommunikation verwendet.\par

Die Beliebtheit von interaktiven Mehrspieler-Spielen, welche durch das einfache Abrufen einer Webseite spielbar sind, steigt von Jahr zu Jahr, und wird zudem durch die seit Beginn 2020 anhaltende Corona-Pandemie weiter gefördert \cite{BGMarket}.\par

In der Regel basieren Browserbasierte Mehrspieler-Spiele auf einer Client-Server-Architektur, wobei der Server die Rolle des bestimmenden Spielleiters übernimmt. Ein weiterer, gut definierter aber für Browserbasierte Spiele selten verwendeter Ansatz für die Vernetzung von Spielern ist die \acf{P2P}-Architektur. Bei dieser existiert kein zentraler Server, die Nutzer (Peers) sind gleichberechtigt und tauschen Daten direkt untereinander aus. Einer der großen Vorteile der Peer-To-Peer Architektur sind dabei die geringeren Datenmengen, welche über einen zentralen Server verwaltet werden müssen. Dies führt zu Kostenersparnissen in Form von weniger Bedarf an Hardware.\par

Beide dieser Netzwerkarchitekturen finden in der Spieleentwicklung Anwendung -- jedoch ist die Nutzung von \acs{P2P}-Netzwerken zum Datenaustausch bei Browserbasierten Mehrspieler-Spielen begrenzt. Dies ist nicht zuletzt auf das Lebensende des Adobe Flash Players im Dezember 2020 zurückzuführen, welcher über Peer-To-Peer Fähigkeiten, ermöglicht durch das Real-Time Media Flow Protocol von Adobe \cite{adobeRFC}, verfügte. Seitdem existiert -- mit Ausnahme von \acs{WebRTC} -- keine alternative Möglichkeit um Nutzer von Webanwendungen, ohne die Nutzung von Plug-Ins oder Drittanbietersoftware, direkt untereinander zu vernetzen.\par

\section{Zielsetzung}
In dieser Arbeit soll ein Brettspiel, sowie sämtliche benötigten Komponenten zum Aufbau von Peer-To-Peer Netzwerken via \acs{WebRTC} prototypisch entworfen, implementiert und aufgesetzt werden. Ziel ist es, darauf basierend die Anwendbarkeit von \acs{WebRTC} für die Entwicklung von Brettspielen im Browser unter Nutzung von Peer-To-Peer Netzwerken zu evaluieren. Dabei soll primär auf Vor- und Nachteile einer Nutzung von Peer-To-Peer Netzwerken via \acs{WebRTC} im Vergleich zu Client-Server Modellen eingegangen werden.

\section{Aufbau der Arbeit}
// TODO: logischerweise als letztes...
