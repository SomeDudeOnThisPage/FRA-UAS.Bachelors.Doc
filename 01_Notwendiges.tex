\newpage
\chapter*{Eidesstattliche Erklärung}
\chaptermark{Eidesstattliche Erklärung}
\addcontentsline{}{}{Eidesstattliche Erklärung}

Hiermit erkläre ich, Robin Buhlmann, dass ich die vorliegende Bachelorarbeit selbstständig und ohne unerlaubte Hilfe angefertigt, andere als die angegebenen Quellen und Hilfsmittel nicht benutzt und die den benutzten Quellen wörtlich oder inhaltlich entnommenen Stellen als solche kenntlich gemacht habe.

Die Arbeit wurde bisher in gleicher oder ähnlicher Form keiner anderen Prüfungsbehörde vorgelegt und auch nicht veröffentlicht.


\vspace{1cm}

Friedrichsdorf, den

\vspace{1cm}

------------------------------------\hspace{6cm}------------------------------------\\
Datum \hspace{9,4cm} Unterschrift

\newpage

\chapter*{Vorwort}
\chaptermark{Vorwort}
\addcontentsline{}{}{Vorwort}
\setlength{\parindent}{0em}
\setlength{\parskip}{1em}

Danke und so!\par

\color{red}
Textabschnitte in ROT sind temporär und müssen neu geschrieben werden, da ich die extremst schlecht formuliert habe.
\color{black}

\begin{displayquote} 
\textit{"
Die wohl absurdeste Art aller Netzwerke sind die Computernetzwerke. Diese Werke werden von ständig rechnenden Computern vernetzt und niemand weiß genau warum sie eigentlich existieren. Wenn man den Gerüchten Glauben schenken darf, dann soll es sich hierbei um werkende Netze handeln die das Arbeiten und das gesellschaftliche Miteinander fordern und fördern sollen. Großen Anteil daran soll ein sogenanntes Internet haben, dass wohl sehr weit verbreitet sein soll. Viele Benutzer des Internets leben allerdings das genetzwerkte Miteinander so sehr aus, dass das normale Miteinander nahezu komplett vernachlässigt wird (vgl. World of Warcraft)."}\par
\hfill
 -- Netzwerke -- www.stupidedia.org
\end{displayquote}

\chapter*{Zusammenfassung}
\chaptermark{Abstract}
\addcontentsline{}{}{Abstract}

In dieser Arbeit wird die Anwendbarkeit von WebRTC Datenkanälen zur Entwicklung von Browserbasierten, Peer-To-Peer Mehrspieler Brettspielen untersucht. Dabei werden insbesondere die Vor- und Nachteile einer Nutzung von WebRTC im Vergleich zu traditionellen Client-Server Infrastrukturen betrachtet. Dabei wird ein prototypisches Brettspiel entworfen, wobei sämtlicher Spielrelevanter Datenverkehr über WebRTC Datenkanäle abgewickelt wird.\par 
// TODO: eng