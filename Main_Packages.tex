%%%%%%%%%%%%%%%%%%%%%%%%%%%%%%%
%Kopf- und Fußzeilendefinition%
%%%%%%%%%%%%%%%%%%%%%%%%%%%%%%%
\usepackage[headsepline,footsepline]{scrlayer-scrpage}	%mit Trennlinien
\pagestyle{scrheadings}					%Seitenstil umdefiniert
\clearscrheadfoot					%Definition leermachen
\renewcommand{\chapterpagestyle}{scrheadings}		%Kapitelseitenstil umdefiniert
\renewcommand{\chaptermark}[1]{\markboth{\ #1}{}}	%Definition Kapitel?berschrift=\leftmark
\ohead{\leftmark}				%äußere Kopfzeile
\chead{}						%mittlere Kopfzeile
\ihead{}						%innere Kopfzeile
\ofoot{\thepage}				%äußere Fußzeile
\cfoot{}						%mittlere Fußzeile
\ifoot{}						%innere Fußzeile

%%%%%%%%%%%%%%%%%%%%%%%%%%%%%%%%%%%%%%%
%%%%%%%deutsche Zeichenkodierung%%%%%%%
%%%%%%%%%%%%%%%%%%%%%%%%%%%%%%%%%%%%%%%
\usepackage[utf8]{inputenc}
\usepackage[english,ngerman]{babel} 

%%%%%%%%%%%%%%%%%%%%%%%%%%%%%%%%%%%%%%%%%%%%%%
%%%%%%%%%%%%Verschiedene Schriftarten%%%%%%%%%
%%%%%%%%%%%%%%%%%%%%%%%%%%%%%%%%%%%%%%%%%%%%%%
%\usepackage{crimson}
%\usepackage[adobe-utopia]{mathdesign}
%\usepackage{libertine}
%\usepackage{lmodern}
\usepackage{kpfonts}
%\usepackage{stix}
%\usepackage{libertinust1math}
\usepackage[T1]{fontenc}
%\renewcommand*\familydefault{\sfdefault}

%%%%%%%%%%%%%%%%%%%%%%%%%%%%%%%%%%%%%%%%%%%%%%
%%%%%%%%%%%%%%%Zusatzpackages%%%%%%%%%%%%%%%%%
%%%%%%%%%%%%%%%%%%%%%%%%%%%%%%%%%%%%%%%%%%%%%%
\usepackage{a4wide}
\usepackage{longtable}
\usepackage{csquotes}
\usepackage{longtable}
\usepackage[locale=DE]{siunitx}
\usepackage{eurosym}					%Fancy ? Zeichen
\usepackage[printonlyused, withpage]{acronym}			%Abk?rzungsverzeichnis
\usepackage{wasysym}					%Symbole und Sonderzeichen
\usepackage{longtable}					%Formelzeichentabelle/lange Tabelle
\usepackage{romannum}					%R?mische Zahlen
\usepackage{wrapfig}					%Bilder nebeneinander/ Text neben Bild
\usepackage{floatflt,epsfig} 				%Bilder im eps-format einf?gen
\usepackage{exceltex}					%Exceldokumente einbinden
\usepackage{multirow}					%Tabellenzellen verbinden
\usepackage{graphicx}					%ich glaube ein Vektorzeichentool
\usepackage[hyphens]{url}				%Websites einbinden
\usepackage{pdfpages} 					%direkt pdf Datein einbinden
\usepackage{amsmath}					%matheumgebung f?r Formeln und Gleichungen
\usepackage{booktabs}					%ich glaube f?r links im PDF
\usepackage{capt-of}					%Captions f?r non-float-Objekte
\usepackage{appendix}					%erm?glicht das erstellen von Anhangsverzeichnissen
\usepackage{listings}
\renewcommand{\lstlistlistingname}{Codeverzeichnis}
\renewcommand{\lstlistingname}{Quellcode}
\usepackage{xcolor}
\usepackage{tikz}					%tikz Paket f?r Grafiken + Definitionen
\usetikzlibrary{arrows,positioning}
\usetikzlibrary{decorations.pathmorphing}
\usetikzlibrary{shapes.geometric}
\usepackage{svg}
\usepackage{caption}
\usepackage{tabularx}

%%%%%%%%%%%%%%%%%%%%%%%%%%%%%%%%%%%%%%%%
%%%%Stil des Literaturverzeichnisses%%%%
%%%%%%%%%%%%%%%%%%%%%%%%%%%%%%%%%%%%%%%%
%\bibliographystyle{abbrvdin}		%sortiert nach Autor 		- nummeriert [1]
%\bibliographystyle{unsrtdin}		%sortiert nach Textmarke	- nummeriert [1]
\bibliographystyle{alphadin}		%sortiert nach Autor		- [AUTOR-JAHR]


%%%%%%%%%%%%%%%%%%%%%%%%%%%%%%%%%%%%%%%%%
%%%%%%%%%%%%%%special Stuff%%%%%%%%%%%%%%
%%%%%%%%%%%%%%%%%%%%%%%%%%%%%%%%%%%%%%%%%
\usepackage[colorlinks,    				%PDF linked Verzeichnis
pdfpagelabels,
pdfstartview = FitH,
bookmarksopen = true,
bookmarksnumbered = true,
linkcolor = black,
plainpages = false,
hypertexnames = false,
citecolor = black,
breaklinks = true,
allcolors = black] {hyperref}  

% custom caption command thingy to cite if needed
% vspace -11 to force effective single spacing between caption and src
\newcommand{\source}[1]{\vspace{-11pt} \caption*{ Quelle: {#1}} }

\input{listing-environments.tex}

%\renewcommand{\labelitemi}{-}   %aufzählungszeichen strich statt punkt - beliebig anpassbar
\definecolor{codegreen}{rgb}{0,0.6,0}
\definecolor{codegray}{rgb}{0.5,0.5,0.5}
\definecolor{codepurple}{rgb}{0.58,0,0.82}
\definecolor{backcolour}{rgb}{0.95,0.95,0.92}

\definecolor{whitecolour}{rgb}{1.00,1.00,1.00}
\definecolor{redcolor}{rgb}{1.00,0.00,0.00}
\definecolor{bluecolor}{rgb}{0.00,0.00,1.00}
\definecolor{darkorange}{rgb}{0.61,0.34,0.00}

\lstdefinestyle{STYLE_COMMAND_LINE_ARGUMENT_SINGLE_LINE}{
    backgroundcolor=\color{whitecolour},
    keywordstyle=\color{backcolour},
    basicstyle=\ttfamily\small,
    breakatwhitespace=false,         
    breaklines=true,                 
    captionpos=b,                    
    keepspaces=true,                 
    numbers=none,                    
    numbersep=0pt,                  
    showspaces=false,                
    showstringspaces=false,
    showtabs=false,                  
    tabsize=2,
    frame=tb,
    xleftmargin=0.5cm,
    xrightmargin=0.5cm,
}

\lstdefinestyle{STYLE_CODE_JS}{
    backgroundcolor=\color{whitecolour},   
    commentstyle=\color{magenta},
    keywordstyle=\color{darkorange},
    numberstyle=\tiny\color{codegray},
    stringstyle=\color{codegray},
    basicstyle=\ttfamily\footnotesize\linespread{-0.8},
    breakatwhitespace=false,         
    breaklines=true,                 
    captionpos=b,                    
    keepspaces=true,                 
    numbers=left,                    
    numbersep=5pt,                  
    showspaces=false,                
    showstringspaces=false,
    showtabs=false,        
    frame=tb,          
    tabsize=2
}

\lstdefinestyle{STYLE_ICE_CANDIDATE_0}{
    backgroundcolor=\color{whitecolour},
    keywordstyle=\color{backcolour},
    basicstyle=\ttfamily\small,
    breakatwhitespace=false,  
    basicstyle=\ttfamily\footnotesize,
    columns=fullflexible,
    escapeinside={<@}{@>},
    breakatwhitespace=false,         
    breaklines=true,                 
    captionpos=b,                    
    keepspaces=true,                 
    numbers=left,                    
    numbersep=5pt,                  
    showspaces=false,                
    showstringspaces=false,
    showtabs=false,  
    frame=tb,                
    tabsize=2
}

\lstdefinelanguage{js}{
  keywords={typeof, new, for, true, false, catch, function, return, null, catch, switch, var, const, let, if, in, while, do, else, case, break},
  keywordstyle=\color{orange}\bfseries,
  ndkeywords={class, default, export, prototype, boolean, throw, implements, import, this},
  ndkeywordstyle=\color{darkorange}\bfseries,
  identifierstyle=\color{black},
  sensitive=false,
  comment=[l]{//},
  %morecomment=[s]{/*}{*/},
  commentstyle=\color{purple}\ttfamily,
  stringstyle=\color{red}\ttfamily,
  morestring=[b]',
  morestring=[b]",
  morestring=[b]`
}

% \newcommand{\ra}[1]{\renewcommand{\arraystretch}{#1}}

% nicht alle wörter breaken...
\tolerance=1
\emergencystretch=\maxdimen
\hyphenpenalty=10000
\hbadness=10000